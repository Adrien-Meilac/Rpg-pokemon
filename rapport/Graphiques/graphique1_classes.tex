\documentclass[11 pt]{article}

\usepackage{amsthm}
\usepackage{amsmath}
\usepackage{amssymb}
\usepackage{mathrsfs}
\usepackage{pgf}
\usepackage{tikz}
\usetikzlibrary{arrows,automata}
\usepackage[utf8]{inputenc}
\usepackage[T1]{fontenc}
\usepackage[francais]{babel}

\begin{document}
\pagestyle{empty}
\begin{center}
\large{\textbf{Graphe des classes}}
\end{center}

\vspace{1 cm}
Prenons par exemple le graphe suivant pour repr�senter les liens entre les diff�rentes classes:

\vspace{1 cm}
\begin{tikzpicture}[->,>=stealth',shorten >=1pt,auto,node distance=2.8cm,
                    semithick]
  \tikzstyle{every state}=[fill=white,draw=black,text=black]

  \node[state]         (A)                    {type};
  \node[state]         (B) [above right of=A] {species};
  \node[state]         (D) [below right of=A] {move};
  \node[state]         (C) [below right of=B] {pokemon};
  \node[state]         (E) [below of=D]       {exp and level};
  \node[state]		   (F) [below left of=A]  {target};
  \node[state]		   (G) [below of=F]       {flag};
  \node[state]		   (H) [below of=G]       {damage};

  \path (A) edge              node {} (B)
            edge              node {} (D)
		(B) edge              node {} (C)
        (D) edge              node {} (C)
        (E) edge              node {} (C)
        (F) edge			  node {} (D)
        (G) edge			  node {} (D)
        (H) edge			  node {} (D);
\end{tikzpicture}
\end{document}