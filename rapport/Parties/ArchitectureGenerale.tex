\section{Architecture générale du code}

D'après GitHub, notre code est composé de 35 fichier .cpp et de 23 fichier .h, et notre projet contient environ 14 000 lignes. Pour ne pas se perdre dans notre projet, nous avons donc essayé de hiérarchiser au maximum notre code et de créer le namespace PKMN pour pouvoir reconnaître plus facilement nos fonctions. 

\begin{figure}[!h]\centering
\subimport{../Graphiques/}{hierarchie_globale_projet.tex}
\caption{\label{ArchiGenerale}Schéma de l'architecture générale du projet}
\end{figure}

L'organigramme \ref{ArchiGenerale} illustre schématiquement les interactions entre les différentes parties de notre code. Lorsque le programme est lancé par l'utilisateur. Ce dernier va commencer par préparer les méthodes d'accès aux données (stockées dans le répertoire Tools) et va remplir la structure de nos données (la plupart des classes que nous avons écrites se trouvent dans Pokémon) avec les données stockées dans les répertoires Data et BackUp. 

Ces structures de données sont majoritairement composées d'accesseurs et permettent de modifier les données et de les sauvegarder via les méthodes d'accès. Nous avons essayé de les différentier des structures du jeu (qui correspond principalement aux fichiers BattleWildPokemon et main). Ces dernières utilisent les structures de données mais n'agissent pas directement sur les fichiers. 

De l'autre côté, l'utilisateur a uniquement accès aux commandes graphiques (stockées dans Graphics). Nous avons décider de les séparer de la structure du jeu pour :
\begin{itemize}
\item éviter d'avoir des fonctions trop longues,
\item  distinguer le cœur de notre code qui est en C++ alors que les fonctions graphiques sont en C
\item pouvoir utiliser les options avancées de classe qui sont incompatibles avec l'option extern C. 
\item avoir des fonctions graphiques qui ne touchent pas aux données et réalisent juste de l'affichage
\end{itemize}
Pour pouvoir réaliser une telle coupure, nous avons créé des fonctions graphiques qui prennent l'écran en paramètre et qui le remplissent. Elles fonctionnent avec des boucles infinies qui vont s’arrêter (soit parce que l'utilisateur a sélectionné un choix, soit spontanément dans le cas d'un effet aléatoire) et renvoyer un ''flag'' (codé sous forme de string ici) aux structures du jeu qui vont agir en fonction du flag reçu. Nous nous sommes servis du double buffering pour que l'utilisateur ne puisse pas voir les jointures entre nos différents affichages. 

Enfin notre programme rédige un log (stocké dans stdout.txt) qui nous permet de tracer les incohérences dans le gameplay. 

Le jeu en lui même est une boucle infinie qui s'arrête lorsque l'utilisateur ferme la fenêtre. 