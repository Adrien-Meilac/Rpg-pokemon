\subsection{Fonctions graphiques réalisées en C (Graphics)}

Graphics est un répertoire contenant toutes les fonctions graphiques gérant l'interface. Nous avons tenu à bien segmenter le code et à créer des fonctions qui se contentent d'afficher mais ne touchent pas à nos arguments. Nous avons fait seulement deux entorses à cette règle par souci de simplicité. Il s'agit de 
 hpallydecrease et hpfoedecrease qui impactent respectivement la barre de vie du Pokemon allié et celle du Pokemon ennemi en combat.
 
Nous avons distingué deux principales natures de graphiques:

\begin{itemize}
\item Field (lié à Field.h), qui permet de générer la carte du jeu sur laquelle se déplace le personnage. Nous avons codé le déplacement du personnage et la rencontre de Pokemon sauvages. A l'aide de la touche Espace, le joueur peut afficher un menu lui permettant notamment de consulter ses Pokemon.
\item les interfaces de combat (lié à Battle.h), comprenant l'écran de combat et les menus associés. Elles sont créées à l'aide de macro \#define permettant de redefinir sur chacune de nos fonctions graphiques les mêmes éléments à quelques variations près. Pour créer une interface de combat, il faut afficher l'arrière plan, la base du Pokemon ennemi, celle du Pokemon allié, le Pokemon ennemi, le Pokemon allié, la barre de menus présente en bas, les deux boites de data contenant la vie des Pokemon, ainsi que leurs nom, leurs niveau (Level), l'image de leurs barre de vie, et enfin la vie sous forme d'un entier pour le Pokemon allié. L'utilisation de \#define nous a donc permis de condenser la déclaration de ces fonctions et de réduire drastiquement leur nombre de ligne. Par exemple, SET\_BATTLE\_DATABOX définit 11 paramètres en une seule ligne.  
\end{itemize}

Pour pouvoir enchaîner les effets visuels, nous avons eu recours au double buffering. En effet, nos fonctions graphiques prennent en paramètre screen qui correspond à l'écran global. Elles vont l'utiliser jusqu'au moment où un flag est émis. La fonction graphique suivante reprend alors la variable screen et la réutilise. Le double buffering permet de ne pas montrer le nouvel appel aux images de fond et l'utilisateur ne voit aucun changement visuel.

Toutes nos fonctions graphiques n'ont pas de rétroaction, c'est pourquoi la majorité sont de type void. D'autre part, celles qui renvoient un flag (les fonctions graphiques non void), peuvent le faire de manière souhaitée (l'utilisateur clique sur une option) ou de manière involontaire (déclenchement d'un combat à l'aide d'un événement aléatoire, comme les combats)